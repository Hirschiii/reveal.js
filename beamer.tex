% Created 2021-12-10 Fr 16:54
% Intended LaTeX compiler: pdflatex
\documentclass[bigger]{beamer}

        \mode<{{{beamermode}}}>

        \usetheme{{{{beamertheme}}}}

        \usecolortheme{{{{beamercolortheme}}}}

        \beamertemplateballitem

        \setbeameroption{show notes}
        \usepackage[utf8]{inputenc}

        \usepackage[T1]{fontenc}

        \usepackage{hyperref}

        \usepackage{color}
        \usepackage{listings}
        \lstset{numbers=none,language=[ISO]C++,tabsize=4,
    frame=single,
    basicstyle=\small,
    showspaces=false,showstringspaces=false,
    showtabs=false,
    keywordstyle=\color{blue}\bfseries,
    commentstyle=\color{red},
    }

        \usepackage{verbatim}

        \institute{{{{beamerinstitute}}}}
          
         \subject{{{{beamersubject}}}}

\usepackage[utf8]{inputenc}
\usepackage[normalem]{ulem}
\usepackage{mathtools}
\author{Niklas von Hirschfeld}
\date{2010-03-30 Tue}
\title{Writing Beamer presentations in org-mode}
\hypersetup{
 pdfauthor={Niklas von Hirschfeld},
 pdftitle={Writing Beamer presentations in org-mode},
 pdfkeywords={},
 pdfsubject={},
 pdfcreator={Emacs 27.2 (Org mode 9.4.4)}, 
 pdflang={Germanb}}
\begin{document}

\maketitle
\tableofcontents


\section{Introduction}
\label{sec:orga324e86}
\begin{frame}[fragile]\frametitle{A simple slide}
\label{sec:org6695a63}
This slide consists of some text with a number of bullet points:

\begin{itemize}
\item the first, very @important@, point!
\item the previous point shows the use of the special markup which
translates to the Beamer specific \emph{alert} command for highlighting
text.
\end{itemize}


The above list could be numbered or any other type of list and may
include sub-lists.
\end{frame}
\end{document}
